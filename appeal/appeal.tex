\documentclass{article}
\usepackage{graphicx} % Required for inserting images
\usepackage{amsmath}
\title{IDL - Appeal, Exam B. }
\author{David Ponarovsky}
\date{August 2025}

    \newcommand{\DXX}[1]{ \frac{d}{ d #1 } }
\begin{document}

\maketitle

\paragraph{ Question 7 - RNN nets.} 
Does a recursive net of type Elman, that gets the zero vector as input at each step, can count? Namely, outputs the value $t$ at its $t$-th step?
\begin{enumerate}
  \item Yes
  \item No
  \item In general No, Yet when given $t$ as initial input, yes. 
\end{enumerate}
\paragraph{ Question 8 - Inception Score. } Which of the following scenarios is expected to yield high IS score, although the generated images are at low quality?
\begin{enumerate}
  \item The model generates a blurred images, yet with high number of categorical.  
  \item The model generates clear elements that are easy to classify, Yet the elements (inside them) are unrealistic. 
  \item The model generates good images and then add them a random noise. 
  \item The model generates images with high variance between the outputs at the pixels level, but their sementic content repeats on itself.    
\end{enumerate}
\paragraph{ Question 14 - VAEs. } What is the reason for the generated images by VAEs been blurred compared to the images generated by GANs ?   
\begin{enumerate}
  \item Usage of reconstruction loss that smooth sharp items.   
  \item  KL-divergence element that impair the disentangle (or separation) of different samples in the latent space. 
  \item Low presentation ability of the VAEs architecture. 
  \item Entering too mach noise into the latent space, which after decoding comes into fact in burred image. 

\end{enumerate}


\end{document}

