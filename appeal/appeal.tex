\documentclass{article}
\usepackage{graphicx} % Required for inserting images
\usepackage{xcolor}
\usepackage{amsmath}
\usepackage{hyperref}
\title{IDL - Appeal, Exam B. }
\author{David Ponarovsky}
\date{August 2025}

    \newcommand{\DXX}[1]{ \frac{d}{ d #1 } }
    \newcommand{\inb}[1]{ \color{blue}#1 \color{black} }
\begin{document}

\maketitle

\paragraph{ Question 7 - RNN nets.} 
Does a recursive net of type Elman, that gets the zero vector as input at each step, can count? Namely, outputs the value $t$ at its $t$-th step?
\begin{enumerate}
  \item Yes
  \item No
  \item In general No, Yet when given $t$ as initial input, yes. 
\end{enumerate}
\inb{ My answer: (3), Correct answer: (1)}. 
%I believe that the confusion emits form the order of entetis, The question: 'Is there an Elman cell that can count until $t$ for an arbitrary $t$?' is a different question than: 'Fix $t$, is there exist an Elman cell that can count until $t$'? 

%For, the second question: There is family of unbound fan-in/out circuits, at width $poly(|t|)$ (the length of the encoding of $t$), that implement addition: \href{https://people.clarkson.edu/~alexis/PCMI/Notes/lectureB02.pdf}{[addition in $AC_{0}$]}. 
%It's not hard to see that the implementation in the notes can realized using Elman cell, and there fore one can find such realization that adds $1$ to the input which is entered via the hidden channel.  

%For the first question, in which we first fix the Elman cell architecture, and therefore also fix the width of the output. Thus the number can be outputted using the cell has to has a suitable length and can not be arbitrary.

%I marked the third option since, saying 'on given $t$ as initial input' implies that $t$ is validate input/token to the cell, namely has length which matches the cells weights.  

I believe that the confusion arises from the order of entities. The question: 'Is there an Elman cell that can count until $t$ for an arbitrary $t$?' is a different question than: 'Fix $t$, does there exist an Elman cell that can count until $t$?'

For the second question: There is a family of unbounded fan-in/out circuits, at width $poly(|t|)$ (the length of the encoding of $t$), that implement addition: \href{https://people.clarkson.edu/~alexis/PCMI/Notes/lectureB02.pdf}{[addition in $AC_{0}$]}. It's not hard to see that the implementation in the notes can be realized using an Elman cell, and therefore one can find such a realization that adds $1$ to the input which is entered via the hidden channel.

For the first question, in which we first fix the Elman cell architecture, and therefore also fix the width of the output. Thus, the number that can be outputted using the cell has to have a suitable length and cannot be arbitrary.

I marked the third option since saying 'on given $t$ as initial input' implies that $t$ is a valid input/token to the cell, namely has a length that matches the cell's weights.


\paragraph{ Question 8 - Inception Score. } Which of the following scenarios is expected to yield high IS score, although the generated images are at low quality?
\begin{enumerate}
  \item The model generates a blurred images, yet with high number of categorical.  
  \item The model generates clear elements that are easy to classify, Yet the elements (inside them) are unrealistic. 
  \item The model generates good images and then add them a random noise. 
  \item The model generates images with high variance between the outputs at the pixels level, but their sementic content repeats on itself.    
\end{enumerate}
\paragraph{ Question 14 - VAEs. } What is the reason for the generated images by VAEs been blurred compared to the images generated by GANs ?   
\begin{enumerate}
  \item Usage of reconstruction loss that smooth sharp items.   
  \item  KL-divergence element that impair the disentangle (or separation) of different samples in the latent space. 
  \item Low presentation ability of the VAEs architecture. 
  \item Entering too mach noise into the latent space, which after decoding comes into fact in burred image. 

\end{enumerate}


\end{document}

